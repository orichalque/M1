\documentclass[a4paper]{article}
\usepackage[francais]{babel}
\usepackage{array}
\usepackage[utf8]{inputenc}
\usepackage[T1]{fontenc}
\usepackage{amssymb}
\newtheorem{prop}{Propriété}
\newcommand{\vabs}[1]{\ensuremath{|#1|}}
\title{Grand tournois d'Ultimate Freesbee}
\author{Jean Dupont}

\begin{document}
\maketitle
Le grand tournoi d'\textit{Ultimate Freesbee} aura lieu cette année à Talon-Plage. La liste des participants (voir table 1) sera distribuée à tous les joueurs au début du premier match.
	
\begin{tabular}{lllr}
	\multicolumn{4}{c}{TAB. 1 - Participants} \\
	\hline 
	\textbf{Nom} & \textbf{Prénom} & \textbf{Adresse} & \textbf{Âge} \\
	\hline 
	Dupont & Jean & 34 rue des oliviers. & 34 \\
	\multicolumn{2}{c}{} & Toto sur Vire & \\
	Durant & Paul & 131, impasse des pommes. 44200 Blabla sur Erdre & 4 \\
	\multicolumn{2}{c}{} & 44200 Blabla sur Erdre & \\
	De Haviland & Marc-Aurèle & 6 place Max Mahon. 44110 & 102 \\
	\multicolumn{2}{c}{} & Nantes & \\
	\hline
\end{tabular}

\begin{prop}[Inégalité triangulaire]. Pour toutes valeurs réelles de $a$, $b$, $c$, on a:
\begin{equation}
\vabs{a - b} \leqslant \vabs{a - c} + \vabs{c - b}. 
\end{equation}
\end{prop}
\end{document}
